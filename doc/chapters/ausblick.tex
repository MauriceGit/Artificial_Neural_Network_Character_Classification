\chapter{Mögliche Weiterentwicklungen}
Bei der Entwicklung des neuronalen Netzes lag der Fokus auf der eigenständigen Erarbeitung von Grundlagen im Bereich des maschinellen Lernens. Durch den Ausbau des Netzes lässt sich dessen Funktionsumfang noch deutlich erweitern. Mögliche Ansatzpunkte für Erweiterungen können dabei aus den folgenden Bereichen gefunden werden

\begin{itemize}
\item Paralellisierung
\item GPU-Computing
\item Hour-Glass (Unterschiedliche Größen der versteckten Schichten)
\item Dropout Verfahren (Zur effizienten Suche von Minima beim Gradientenvefahren)
\item Simulated Annealing
\item Einsatz verschiedener Aktivierungsfunktionen
\end{itemize}

Weiterhin ist die Entwicklung eines Statistik-Moduls sinnvoll um die Ergebnisse aus den Lernvorgängen durch automatisch generierte Plots und Datensätzen zu ermöglichen.