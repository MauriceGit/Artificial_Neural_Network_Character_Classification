\chapter{Einleitung}
Das Projekt \emph{Learning und Softcomuting} setzt sich mit dem Thema \emph{Lernversuche zur automatischen Zeichenerkennung mit einem Backpropagation-Netz} auseinander. Die Thematik der neuronalen Netze mit Backpropagation-Lern-Algorithmus soll dabei von den Teilnehmern durchdrungen und durch den Einsatz eines neuronalen Netzes umgesetzt werden. Ziel ist es, das Netzwerk so zu realisieren und zu trainieren, dass es die 26 Großbuchstaben des Alphabets erkennt. 

Grundsätzliche handelt es sich bei diesem Projekt um ein Gebiet des maschinellen Lernens. Vereinfacht gesagt, ist es die Aufgabe eines Computerprogramms durch Erfahrung zu lernen und somit eine Aufgabe möglichst optimal zu lösen.

Anstelle eine fertigen Frameworks wie \emph{PyBrain} \footnote{http://pybrain.org/} oder \emph{scikitLearn} \footnote{http://scikit-learn.org/stable/} fokussiert sich dieses Dokument auf die Konzeption und Implementierung eines eigenen neuronalen Netzes in Python. 