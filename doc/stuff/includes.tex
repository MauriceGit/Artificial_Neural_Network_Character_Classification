% **********************************************
% Jens Begemann INF 101419
% inf101419@fh-wedel.de / jensbegemann@posteo.de
% **********************************************

% ----------------------------------------------
% Typ des Dokumentes festlegen
% ----------------------------------------------
% Bindekorrektur (BCOR), Bib und alle Listen mit im Inhalt (totoc)
\documentclass[12pt, a4paper, BCOR=5mm, bibliography=totoc, listof=totoc, headsepline, headinclude=false, footinclude=false] {scrreprt}

% Satzspiegel mit KOMA-Skript
\usepackage[automark]{scrpage2} 

% Deutsche Sprache mit umlauten
\usepackage[ngerman]{babel}
\usepackage[utf8x]{inputenc}

% Schriftart auf Modern setzen für bessre PDF Lesbarkeit
\usepackage[T1]{fontenc}
\usepackage{lmodern}

% Kopf und Fußzeilen löschen
\clearscrheadings
\clearscrplain
\clearscrheadfoot

% Automatische Kopfzeile mit Kapitel und Abschnittsnamen
\automark[section]{chapter}

% Unten rechts Seitenzahl ausgeben
\ofoot[\pagemark]{\pagemark}

% Kopfzeile nur auf Seiten ohne neuen Kapitelanfang
\ihead[]{\leftmark}
\ohead[]{\rightmark}

% Kopfzeile aktivieren
\pagestyle{scrheadings}

% Anpassung der Schriftgröße der Captions
\setkomafont{caption}{\small} % kleiner  als die normale Schrift
\setkomafont{captionlabel}{\usekomafont{caption}}

% Entfernt Einrückung beim Absatz
%\setlength{\parskip}{1em}
\setlength{\parindent}{0em}
% Groessere Abstand zwischen Absaetzen
\setlength{\parskip}{1.5ex}

% ----------------------------------------------
% Konstanten & Befehle
% ----------------------------------------------
% Mehr Horizontaler Abstand in Tabellen
\renewcommand{\arraystretch}{1.2}

% Schusterjungen & Hurenkinder verhindern
\clubpenalty = 10000
\widowpenalty = 10000 
\displaywidowpenalty = 10000

% Referenzen
%\newcommand{\imgref}[1]{(Abbildung~\ref{#1},~S.~\pageref{#1})}
%\newcommand{\tabref}[1]{(Tabelle~\ref{#1},~S.~\pageref{#1})}
\newcommand{\imgref}[1]{(Abbildung~\ref{#1})}
\newcommand{\tabref}[1]{(Tabelle~\ref{#1})}
\newcommand{\lstref}[1]{(Listing~\ref{#1})}
\newcommand{\chapref}[1]{(Kapitel~\ref{#1})}
\newcommand{\appref}[1]{(Anhang~\ref{#1})}

% ----------------------------------------------
% Grafiken, Bilder, Tabellen, etc..
% Grafikpaket einbinden um Bilder zu implementieren
% ----------------------------------------------
\usepackage{microtype}       			% optischer Randausgleich, falls pdflatex verwandt
\usepackage{graphicx}					% Grafiken in pdfLaTeX
\DeclareGraphicsExtensions{.pdf,.bmp,.png,.mps} 
\usepackage{pdfpages}					% PDF Dateien einbinden
\usepackage[printonlyused]{acronym} 	% list of acronyms and abbreviations
\usepackage{multicol}               	% Mehrspaltige Bereiche
\usepackage{setspace}					% Setzen des Zeilenabstandes
\usepackage{lscape}						% Setzen des Zeilenabstandes
\usepackage[disable]{todonotes}					% todo Notes und missingfigures
\usepackage{textcomp}         			% Symbole (wie griechisches Alphabet)
\usepackage{epstopdf}
\usepackage{amssymb}					% Mathesymbole
\usepackage{amsmath}                
\usepackage{engtlc}                     % Kbit\s in Si Schreibweise usw...
\usepackage{longtable}
\usepackage{multirow}                   % Zusammengefasste Tabellen Spalten
\usepackage{url}
\usepackage{siunitx}                    % Maßeinheiten fuer SI system
\usepackage[nostamp]{draftwatermark} %nostamp             % Wasserzeichen
\SetWatermarkScale{0.8}
\SetWatermarkLightness{0.9}
\SetWatermarkText{draft}

\usepackage{hyperref}                    % Verlinkung im PDF für Kapitel etc.
\hypersetup{
    colorlinks,
    citecolor=black,
    filecolor=black,
    linkcolor=black,
    urlcolor=black
}
%\hypersetup{linktocpage}


% ----------------------------------------------
% Codedarstellung mit Syntaxhighlighting
% ----------------------------------------------
\usepackage{xcolor}					% Farbe fuer Syntaxhervorhebung
\usepackage{listings}				% Quellcode darstellung
\definecolor{dkgreen}{rgb}{0,0.6,0}
\definecolor{gray}{rgb}{0.5,0.5,0.5}
\definecolor{mauve}{rgb}{0.58,0,0.82}

\lstset {
	language = C,
	tabsize = 2,
	frame = single,
	backgroundcolor = \color{white},
	numbers = left,
	numberstyle = \color{gray},
	numberblanklines = false,
	commentstyle=\color{dkgreen},
	identifierstyle = \color{black},
	stringstyle=\color{mauve},
 	basicstyle = \ttfamily \color{black} \scriptsize,
 	keywordstyle=\bf\color{blue},
 	%inputencoding = utf8/latin1,
 	showstringspaces = false,
 	%xleftmargin = \parindent,
 	xleftmargin = 5ex,
 	captionpos = t,
}

% ----------------------------------------------
% Hacks ...
% ----------------------------------------------



% ----------------------------------------------
% EOF
% ----------------------------------------------